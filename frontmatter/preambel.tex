%!TEX TS-program = pdflatexmk
%!TEX root = ../handbook.tex

% Copyright 2018 RailToolKit (Attribution 4.0 International, CC-BY 4.0)
% You are free to copy and redistribute the material in any medium or format. You are free to remix, transform, and build upon the material for any purpose, even commercially. You must give appropriate credit, provide a link to the license, and indicate if changes were made. You may not apply legal terms or technological measures that legally restrict others from doing anything the license permits. No warranties are given.
\chapter*{\IfLanguage{english}{Aim and Materials}\IfLanguage{ngerman}{Ziel und Materialien}}

\noindent
\IfLanguage{english}{The aim of the learning game is to simulate and experience the driving dynamics of trains in the context of block division. This requires:}
\IfLanguage{ngerman}{Ziel des Lernspieles ist die Fahrdynamik von Zügen im Zusammenhang mit Blockteilung zu simulieren und zu erfahren. Dafür wird benötigt:}
\begin{itemize}
  \IfLanguage{english}{
    \item two trains with different driving dynamics
    \item a track consisting of spaces
    \item train berths
    \item signals for block division
    \item if necessary, turnouts
  }
  \IfLanguage{ngerman}{
    \item zwei Züge mit unterschiedlicher Fahrdynamik
    \item eine Strecke, bestehend aus Spielfeldern
    \item Halteplätze für die Züge
    \item Signale für die Blockteilung
    \item ggf. Weichen
  }
\end{itemize}
\IfLanguage{english}{Real continuous dimensions time ($t$) and distance ($s$) are assigned to discrete units of rounds ($t$) and spaces ($s$).
Thus, the simulation is round-based in order to imitate the steps of a computer.}
\IfLanguage{ngerman}{Reale kontinuierliche Größen Zeit ($t$) und Strecke ($s$) werden dabei in diskrete Einheiten von Runden ($t$) und Felder ($s$) eingeteilt.
Die Simulation erfolgt also Rundenbasiert, um im Schrittverfahren einen Computer nachzuahmen.}

\vspace*{\fill}

\IfLanguage{english}{
  {\noindent\large Version \vhCurrentVersion\ from \vhCurrentDate } \\[0.3cm]
  \ccLogo \ccAttribution ~This work is licensed under a Creative Commons License (CC BY 4.0).
}
\IfLanguage{ngerman}{
  {\noindent\large Version \vhCurrentVersion\ vom \vhCurrentDate } \\[0.3cm]
  \ccLogo \ccAttribution ~Dieses Werk steht unter der Creative Commons Lizens (CC BY 4.0).
}