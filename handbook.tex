% Hinweis:
% Alle Einstellungen, die dem Briefschreiber und seinem Institut zugeordnet sind,
% stehen jeweils in einer eigenen .lco-Datei.
% Der Brief lädt dann die Autorendatei, welche wiederum die Institutsdatei lädt
\documentclass[
    nexus,      % Schriftart Arial wählen
    % mono ,    % Darstellung in schwarz -weiß
    10pt,        % gewählte Schriftgröße (standard)
    oneside,
    draft=true,
    green,
    % paper=a5, paper=landscape,
]{tubsbook} % class from http://tubslatex.ejoerns.de/

%!tex root=handbook.tex
% ===============================
%    Kodierung und Sprache
% ===============================
% \usepackage[T1]{fontenc} % ermoeglicht die Silbentrennung von Woertern mit Umlauten
\usepackage{ucs}
\usepackage[utf8x]{inputenc}
\usepackage[ngerman]{babel}
\usepackage[ngerman]{translator}

% -----[ revision history ]----------
\usepackage{vhistory}
% -----[ Creative Commons License ]----------
\usepackage[scale=1.5]{ccicons}

% ===============================
%    Tabellen und Abbildungen
% ===============================
\usepackage{graphicx} % ermoeglicht einbinden von Graphiken
\usepackage{subfigure} % ermoeglicht Bilder nebeneinander aufzureihen
\usepackage{tabularx} % ermoeglicht seitenbreite Tabellen
\usepackage{booktabs} % schoenere Tabellen
\usepackage{multirow}
%\usepackage{longtable} % ermoeglicht Tabellen mit Seitenumbruch
\usepackage{amsmath}
\usepackage{xfrac} % ermoeglich schräg gestellte Brüche mit \sfrac{}{}
\usepackage[locale=DE]{siunitx} % for SI-Units
\sisetup{
  per-mode=fraction,
  fraction-function=\sfrac
}
\usepackage{wasysym} % \permil
\usepackage{enumerate} % Nummeriung aendern
\usepackage{listings}
\usepackage{rotating}
\usepackage{tikz}
% \usepackage{framed} % Rahmen zeichnen

% ===============================
%    chapter/section numbering
% ===============================
\newcounter{question}
\newcommand{\aufgabe}{\stepcounter{question}\bigskip\par\noindent{\normalfont\large\bfseries Aufgabe \arabic{question}}\medskip\par\noindent}
\makeatletter\@addtoreset{chapter}{part}\makeatother%


\logo{\includegraphics[height=\tubslogoHeight]{instituts_logo.pdf}}

\title{Verkehrsfluss bei Bahnsystemen}
\subtitle{Lernspiel}
\author{}
\date{}

\begin{document}
\maketitle[image,logo=right]

\section*{Ziel und Materialien}
  \noindent Ziel des Lernspieles ist die Fahrdynamik von Zügen im Zusammenhang mit Blockteilung zu simulieren und zu erfahren.
  Dafür wird benötigt:
  \begin{itemize}
    \item zwei Züge mit unterschiedlicher Fahrdynamik
    \item eine Strecke, bestehend aus Spielfeldern
    \item Halteplätze für die Züge
    \item Signale für die Blockteilung
    \item ggf. Weichen
  \end{itemize}
  Reale kontinuierliche Größen Zeit ($t$) und Strecke ($s$) werden dabei in diskrete Einheiten von Runden ($t$) und Felder ($s$) eingeteilt.
  Die Simulation erfolgt also Rundenbasiert, um im Schrittverfahren einen Computer nachzuahmen.
  % Eine Runde ist eine Zeiteinheit, in der sich Züge um eine Distanz/Felder fortbewegen.
  \vfill
  {\noindent\large Version \vhCurrentVersion\ vom \vhCurrentDate } \\[0.5cm]
  \ccLogo \ccAttribution ~Dieses Werk steht unter der Creative Commons Lizens (CC BY 4.0).

\newpage
  \newcommand{\MS}{Martin Scheidt}
  \newcommand{\FN}{Felix Nebel}
  \newcommand{\LG}{Lukas Gruber}
  \begin{versionhistory}
    %\vhEntry{<Version>}{<Date>}{<Authors>}{<Changes>}
    \vhEntry{0.1}{17.04.2018}{MS|FN|LG}{ersten Prototyp mit Fahrdynamik erstellt}
    \vhEntry{0.2}{15.05.2018}{MS|LG}{Lehrspiel mit Blocklogik erweitert}
    \vhEntry{0.3}{03.09.2018}{MS}{Handbuch erstellt}
  \end{versionhistory}
  \vhListAllAuthorsLongWithAbbrev
  \vfill
  \tableofcontents

\part{Anleitung}
\chapter{Aufbau der Strecke}
  \noindent Die Strecke besteht aus einer beliebigen Anzahl von Feldern. An die Strecke können Signale oder Bahnsteigkanten angeordnet werden.\\[0.5cm]
  \begin{center}
  \begin{tikzpicture}[>=latex]
    \coordinate (base) at (0,0);
    \draw [dashed] ([shift={(base)}] 0,0) rectangle ++(1,0.5);
    \draw [dashed] ([shift={(base)}] 15,0) rectangle ++(1,0.5);
    \foreach \x in {1,...,14}
      \draw ([shift={(base)}] \x,0) rectangle ++(1,0.5);
    \draw [<->] ([shift={(base)}] 10.5,0.9) -- ++ (4,0) node [fill=white,midway] {Zugbewegung};
    \draw ([shift={(base)}] 5,1.1) -- ++(0,-0.4) -- ++ (3,0) -- ++(0,0.4);
    \draw ([shift={(base)}] 5,0.8) -- ++ (3,0) node [above,midway] {Bahnsteig};
    % distant signal
    \draw [dashed] ([shift={(base)}] 2,-0.1) -- ++(0,-0.2) -- ++(1,0) -- ++(0,0.2);
    \path[draw, line width=1pt] ([shift={(base)}] 2,-0.5) -- ++(0,-0.2);
    \path[draw, line width=1pt] ([shift={(base)}] 2,-0.6) -- ++(0.5,0);
    \path[draw, fill] ([shift={(base)}] 2.5,-0.75) rectangle ++(0.2,0.3);
    \path[draw, fill] ([shift={(base)}] 2.5,-0.45) arc (90:270:0.15);
    \draw [->] ([shift={(base)}] 5.0,-0.5) -- ([shift={(base)}] 2.9,-0.6) node [near start,align=right,fill=white] {Vorsignal};
    % main signal
    \draw [dashed] ([shift={(base)}] 9,-0.1) -- ++(0,-0.2) -- ++(1,0) -- ++(0,0.2);
    \path[draw, line width=1pt] ([shift={(base)}] 9,-0.5) -- ++(0,-0.2);
    \path[draw, line width=1pt] ([shift={(base)}] 9,-0.6) -- ++(0.5,0);
    \path[draw, fill] ([shift={(base)}] 9.5,-0.75) rectangle ++(0.2,0.3);
    \path[draw, fill] ([shift={(base)}] 9.7,-0.75) arc (90:270:-0.15);
    \draw [->] ([shift={(base)}] 6.5,-1.2) -- ([shift={(base)}] 8.9,-0.6) node [near start,align=right,fill=white] {Hauptsignal};
    % axle counter
    \path[draw, line width=1pt] ([shift={(base)}] 9.85,-0.4) -- ++(0.3,0);
    \path[draw, line width=1pt, fill] ([shift={(base)}]  9.94,-0.4) circle (0.04);
    \path[draw, line width=1pt, fill] ([shift={(base)}] 10.06,-0.4) circle (0.04);
    \draw [->] ([shift={(base)}] 13.5,-0.8) -- ([shift={(base)}] 10.2,-0.45) node [near start,align=right,fill=white] {Gleisfreimeldegrenze};
  \end{tikzpicture}
  \end{center}

\chapter{Signalbegriffe}

  \noindent Das Hauptsignal darf nur Fahrt zeigen, wenn im nachfolgenden Blockabschnitt (von Gleisfreimeldegrenze zu Gleisfreimeldegrenze) kein Zug ist.
  Das Vorsignal steht im Bremswegabstand vor dem Hauptsignal und spiegelt den Signalbegriff vom Hauptsignal.
  
  \begin{center}
  \begin{tabular}{rcc}
  \toprule
  & Halt & Fahrt \\
  \hline
  Hauptsignal &
  \begin{tikzpicture}[anchor=base,baseline=-3]
    \draw [fill=red] (0,0) circle (0.3);
    \draw (-0.3,0) -- (0.3,0);
    \path (-0.5,-0.5) rectangle ++(1,1); % background rectangle to unify every cell containing a symbol
  \end{tikzpicture} &
  \begin{tikzpicture}[anchor=base,baseline=-3]
    \draw [fill=green] (0,0) circle (0.3);
    \draw (0,-0.3) -- (0,0.3);
    \path (-0.5,-0.5) rectangle ++(1,1); % background rectangle to unify every cell containing a symbol
  \end{tikzpicture} \\
  Vorsignal   &
  \begin{tikzpicture}[anchor=base,baseline=-3]
    \draw [fill=yellow] (0,0) circle (0.3);
    \draw (-0.22,-0.22) -- ++(0.44,0.44);
    \path (-0.5,-0.5) rectangle ++(1,1); % background rectangle to unify every cell containing a symbol
  \end{tikzpicture} &
  \begin{tikzpicture}[anchor=base,baseline=-3]
    \draw [fill=green] (0,0) circle (0.3);
    \draw (0,-0.3) -- (0,0.3);
    \path (-0.5,-0.5) rectangle ++(1,1); % background rectangle to unify every cell containing a symbol
  \end{tikzpicture} \\
  \bottomrule
  \end{tabular}
  \end{center}

\chapter{Simulation der Fahrdynamik}
  \noindent Die Simulation erfolgt Rundenbasiert.\\ Jede Runde besteht aus mindestens zwei aufeinanderfolgenden Schritten:
  \begin{enumerate}
    \item Der Zug macht die Bewegung, die in der vorhergehenden Runde gesetzt wurde.
    \item Alle Signale werden entsprechend der Gleisbelegung eingestellt.
    \item (optional) Eine neue Schalthebelposition kann gewählt werden.
  \end{enumerate}
   Beim Start des Spiels steht der Zug. Schalthebelposition beginnen bei dem Feld ``$0$''.
   Die unterschiedliche Fahrdynamik von Zügen wird durch unterschiedliche mögliche Schalthebelbewegung abgebildet.

\newpage
\chapter{Beispiel}
  Der Schalthebel wird entlang der grünen Pfeile bewegt. Schalthebelpositionen für einen Personenzug:
  \begin{center}
  \begin{tikzpicture}[scale=0.8]
    \coordinate (base) at (0,0);
    % arrows
    \draw [->,>=latex,line width=1.5pt,green] (1.5,0)  .. controls ( 2,-0.5) and ( 4,-0.5) .. ( 4.5,0);
    \draw [->,>=latex,line width=1.5pt,green] (5.5,0)  .. controls ( 6,-0.5) and ( 8,-0.5) .. ( 8.5,0);
    \draw [->,>=latex,line width=1.5pt,green] (9.5,0)  .. controls (10,-0.5) and (12,-0.5) .. (12.5,0);
    \draw [<-,>=latex,line width=1.5pt,green] (1.5,2)  .. controls ( 2,2.5) and ( 4,2.5) .. ( 4.5,2);
    \draw [<-,>=latex,line width=1.5pt,green] (5.5,2)  .. controls ( 6,2.5) and ( 8,2.5) .. ( 8.5,2);
    \draw [<-,>=latex,line width=1.5pt,green] (9.5,2)  .. controls (10,2.5) and (12,2.5) .. (12.5,2);
    \foreach \x in {0,4,8,12}
      \draw [fill=white] ([shift={(base)}] \x,0) rectangle ++(2,2);
    \node [align=center] at ( 1,1.6) {\si{\kilo\metre\per\hour}};
    \node [align=center] at ( 1,1  ) {\Large $0$};
    \node [align=center] at ( 1,0.3) {\color{blue} $0$ Felder};
    \node [align=center] at ( 5,1.6) {\si{\kilo\metre\per\hour}};
    \node [align=center] at ( 5,1  ) {\Large $40$};
    \node [align=center] at ( 5,0.3) {\color{blue} $1$ Feld};
    \node [align=center] at ( 9,1.6) {\si{\kilo\metre\per\hour}};
    \node [align=center] at ( 9,1  ) {\Large $80$};
    \node [align=center] at ( 9,0.3) {\color{blue} $2$ Felder};
    \node [align=center] at (13,1.6) {\si{\kilo\metre\per\hour}};
    \node [align=center] at (13,1  ) {\Large $120$};
    \node [align=center] at (13,0.3) {\color{blue} $3$ Felder};
  \end{tikzpicture}
  \end{center}
  Ein Zug soll auf folgender Strecke beschleunigen:
  \begin{center}
  \begin{tikzpicture}[>=latex]
    \coordinate (base) at (0,0);
    \draw [dashed] ([shift={(base)}] 10,0) rectangle ++(1,0.5);
    \foreach \x in {-2,...,9}
      \draw ([shift={(base)}] \x,0) rectangle ++(1,0.5) node [below=18pt,left,align=right] {\tiny \x};
    \draw ([shift={(base)}] -2,1.1) -- ++(0,-0.4) -- ++ (3,0) -- ++(0,0.4);
    \draw ([shift={(base)}] -2,0.8) -- ++ (3,0) node [above,midway] {Bf A};
    \path[draw, line width=1pt, fill=white] (0.9,0.25) -- ++(-0.2,-0.2) -- ++(-2.6,0) -- ++(0,0.4) -- ++(2.6,0) -- cycle; % train
    \draw [->] ([shift={(base)}] 5,1.4) -- ([shift={(base)}] 0.85,0.35) node [near start,align=right,fill=white] {Zug (Fahrtrichtung rechts)};
  \end{tikzpicture}
  \end{center}
  Für den Verlauf des Lernspiels beginnen wir mit der ersten Runde und befolgen den Ablauf aus vorheringen Abschnitt \emph{Simulation der Fahrdynamik}.
  In der ersten Runde steht der Zug aus der Anfangsbedingung. Im Beispiel gibt es keine Signale die eingestellt werden können.
  Wir können den Schalthebel um eine Position auf \SI{40}{\kilo\metre\per\hour} nach vorne bewegen. Die Runde ist beendet.

  Die zweite Runde beginnt mit dem Ausführen der Bewegung um ein Feld nach rechts.
  Der Schalthebel kann wieder weiterbewegt und die Runde ist damit beendent.
  Die dritte Runde beginnt mit dem Ausführen der Bewegung um \emph{zwei} Felder nach rechts.
  Der Schalthebel kann wieder weiterbewegt und die Runde ist damit beendent.
  Und so weiter, bis man auf der Position auf \SI{120}{\kilo\metre\per\hour} angekommen ist und der Zug sich gleichmäßig mit drei Felder pro Runde weiter bewegt.\\[0.5cm]
  \begin{tabular}{rcccl}
  \toprule
  Runde & aktuelle   & 1.Schritt & 2.Schritt   & \\
        & Geschwind- & Bewegen   & Schalthebel & \\
        & igkeit     & um        & auf         & \\
  \hline
  $1$   & \SI{0}{\kilo\metre\per\hour} & $0$ Felder & \SI{40}{\kilo\metre\per\hour} &
  \begin{tikzpicture}[scale=0.6]
    \coordinate (base) at (0,0);
    \draw [dashed] ([shift={(base)}] 10,0) rectangle ++(1,0.5);
    \foreach \x in {-2,...,9}
      \draw ([shift={(base)}] \x,0) rectangle ++(1,0.5) node [above=5pt,left,align=right] {\tiny \x};
    \path[draw, line width=1pt, fill=white] (0.9,0.25) -- ++(-0.2,-0.2) -- ++(-2.6,0) -- ++(0,0.4) -- ++(2.6,0) -- cycle; % train
  \end{tikzpicture} \\
  $2$   & \SI{40}{\kilo\metre\per\hour} & $1$ Felder & \SI{80}{\kilo\metre\per\hour} &
  \begin{tikzpicture}[scale=0.6]
    \coordinate (base) at (0,0);
    \draw [dashed] ([shift={(base)}] 10,0) rectangle ++(1,0.5);
    \foreach \x in {-2,...,9}
      \draw ([shift={(base)}] \x,0) rectangle ++(1,0.5) node [above=5pt,left,align=right] {\tiny \x};
    \path[draw, line width=1pt, fill=white] (1.9,0.25) -- ++(-0.2,-0.2) -- ++(-2.6,0) -- ++(0,0.4) -- ++(2.6,0) -- cycle; % train
  \end{tikzpicture} \\
  $3$   & \SI{80}{\kilo\metre\per\hour} & $2$ Felder & \SI{120}{\kilo\metre\per\hour} &
  \begin{tikzpicture}[scale=0.6]
    \coordinate (base) at (0,0);
    \draw [dashed] ([shift={(base)}] 10,0) rectangle ++(1,0.5);
    \foreach \x in {-2,...,9}
      \draw ([shift={(base)}] \x,0) rectangle ++(1,0.5) node [above=5pt,left,align=right] {\tiny \x};
    \path[draw, line width=1pt, fill=white] (3.9,0.25) -- ++(-0.2,-0.2) -- ++(-2.6,0) -- ++(0,0.4) -- ++(2.6,0) -- cycle; % train
  \end{tikzpicture} \\
  $4$   & \SI{120}{\kilo\metre\per\hour} & $3$ Felder & \SI{120}{\kilo\metre\per\hour} &
  \begin{tikzpicture}[scale=0.6]
    \coordinate (base) at (0,0);
    \draw [dashed] ([shift={(base)}] 10,0) rectangle ++(1,0.5);
    \foreach \x in {-2,...,9}
      \draw ([shift={(base)}] \x,0) rectangle ++(1,0.5) node [above=5pt,left,align=right] {\tiny \x};
    \path[draw, line width=1pt, fill=white] (6.9,0.25) -- ++(-0.2,-0.2) -- ++(-2.6,0) -- ++(0,0.4) -- ++(2.6,0) -- cycle; % train
  \end{tikzpicture} \\
  $5$   & \SI{120}{\kilo\metre\per\hour} & $3$ Felder & \SI{120}{\kilo\metre\per\hour} &
  \begin{tikzpicture}[scale=0.6]
    \coordinate (base) at (0,0);
    \draw [dashed] ([shift={(base)}] 10,0) rectangle ++(1,0.5);
    \foreach \x in {-2,...,9}
      \draw ([shift={(base)}] \x,0) rectangle ++(1,0.5) node [above=5pt,left,align=right] {\tiny \x};
    \path[draw, line width=1pt, fill=white] (9.9,0.25) -- ++(-0.2,-0.2) -- ++(-2.6,0) -- ++(0,0.4) -- ++(2.6,0) -- cycle; % train
  \end{tikzpicture} \\
  etc. & & & & \\
  \bottomrule
  \end{tabular}

\part{Ablauf}
  \chapter{Stufe}
  \section{Einführung Fahrdynamik}
    \subsection*{Ausgangssituation}
    \begin{itemize}
      \item ein Zug,
      \item Strecke mit Feldern $-2$ bis $37$,
      \item Bahnsteig A am Feld $-2$ bis $0$,
      \item Bahnsteig B am Feld $14$ bis $16$,
      \item Bahnsteig C am Feld $35$ bis $37$.
    \end{itemize}
    \aufgabe
    Der Zug (auf Feld $0$ in Richtung $37$) steht und hat seinen Schalthebel auf \SI{0}{\kilo\metre\per\hour}.
    \begin{enumerate}[a)]
      \item Wenn der Zug maximal beschleunigt, bis zu welchen Feld gelangt er in \emph{neun} Runden?
      \item Wie viele Runden benötigt man minimal, wenn der Zug in jedem Bahnhof halten soll?
    \end{enumerate}
    Notiere die Lösungschritte in einem Protokoll!
    \begin{center}
      Beispiel für ein Protokoll:\\
      \begin{tabular}{cccc|c}
        \toprule
        Runde   & aktuelle                    & (1. Schritt)& aktuelle          & (2. Schritt)                  \\
                & Geschwindigkeit             & Bewegen um  & Position Zugspitze& Schalthebel auf               \\
        \hline
        $1$     & \SI{0}{\kilo\metre\per\hour}& $0$ Felder  & Feld $0$          & \SI{40}{\kilo\metre\per\hour} \\
        \hline
        $2$     & $\cdots$                    &             &                   &                               \\
        \hline
        $\vdots$&                             &             &                   &                               \\
        \hline 
                &                             &             &                   &                               \\
        \bottomrule
      \end{tabular}
    \end{center}
    \aufgabe
    Der Zug (auf Feld $0$ in Richtung $37$) fährt gerade durch den ersten Bahnhof durch und hat seinen Schalthebel auf der maximalen Geschwindigkeit. 
    \begin{enumerate}[a)]
      \item Wie viele Felder braucht der Zug, bis er zum Stehen gekommen ist?
      \item Wie viele Runden benötigt man, wenn der Zug ohne Halt die Strecke vollständig verlassen soll?
    \end{enumerate}

  \newpage
  \section{Sicht- und Bremsweg}
    \subsection*{Ausgangssituation}

    Unbekannte Strecke mit verschiedenen Sichtverhältnissen:\\[0.5cm]
    \begin{tabular}{rl}
      \toprule
      Sichtverhältnis & Sicht in Feldern \\
      \hline
      Sehr gut & 3 \\
      Normal   & 2 \\
      Schlecht & 1 \\
      \bottomrule
    \end{tabular}
    \aufgabe
    \begin{enumerate}[a)]
      \item Wie schnell kann der Zug bei sehr gutem Sichtverhältnis maximal Fahren um vor einem Hindernis rechtzeitig anzuhalten?
      \item Wie viele Runden benötigt man minimal, um gefahrlos bei normalen Sichtverhältnissen in einem $14$ Felder entfernten Bahnhof zu gelangen?
      \item Wie viele Felder weit müsste man sehen können, um \SI{160}{\kilo\metre\per\hour} fahren zu können?
    \end{enumerate}

  \chapter{Stufe}
  \section{Blockteilung}
    \subsection*{Ausgangssituation}
    Ein Zug und eine beliebig lange Strecke, mit mindestens 3 vollständigen Blöcken.
    Ein Block besteht aus: Sichtpunkt, Vorsignal, Hauptsignal, Signalzugschlussstelle und Räumweg.
    \aufgabe
    \begin{enumerate}[a)]
      \item Platziere die Vor- und Hauptsignale mindestens so, dass \SI{160}{\kilo\metre\per\hour} gefahren werden kann und schlechte Sichtverhältnisse nicht zur Beeinträchtigung führt!
      \item Wie viele Runden ist ein Block mit einer Zugfahrt belegt (vollständige Sperrzeit)?
    \end{enumerate}

  \newpage
  \section{Verkehrsfluss}
    \subsection*{Ausgangssituation}
    \begin{itemize}
      \item Zwei verschiedene Züge mit unterschiedlicher Fahrdynamik.
      \item Beliebige Länge der Strecke, mit mindestens 3 vollständigen Blöcken.
      \item Am Anfang und Ende der Strecke kann jeweils ein Bahnhof mit Weichen angeordnet werden oder die Strecke im Kreis geführt werden.
    \end{itemize}
    \aufgabe
    \begin{enumerate}[a)]
      \item Wie viele Runden benötigt man, wenn beide Züge behinderungsfrei fahren sollen und der schnelle vor dem langsamen Zug fährt?
      \item Wie viele Runden benötigt man, wenn beide Züge behinderungsfrei fahren sollen und der langsame vor dem schnellen Zug fährt?
    \end{enumerate}

\end{document}