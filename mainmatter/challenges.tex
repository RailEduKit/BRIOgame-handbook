%!TEX TS-program = pdflatexmk
%!TEX root = ../handbook.tex

% Copyright 2018 RailToolKit (Attribution 4.0 International, CC-BY 4.0)
% You are free to copy and redistribute the material in any medium or format. You are free to remix, transform, and build upon the material for any purpose, even commercially. You must give appropriate credit, provide a link to the license, and indicate if changes were made. You may not apply legal terms or technological measures that legally restrict others from doing anything the license permits. No warranties are given.

\part{\IfLanguage{english}{Challenges}\IfLanguage{ngerman}{Aufgaben}}
  
\chapter{\IfLanguage{english}{First Stage}\IfLanguage{ngerman}{Erste Stufe}}
  \section{\IfLanguage{english}{Introduction to Driving Dynamics}\IfLanguage{ngerman}{Einführung Fahrdynamik}}
    \setup
      \begin{itemize}
        \IfLanguage{english}{
          \item A single train,
          \item Line with fields    $-2$ to $37$,
          \item Platform A at field $-2$ to $0$,
          \item Platform B at field $14$ to $16$,
          \item Platform C at field $35$ to $37$.
        }
        \IfLanguage{ngerman}{
          \item ein Zug,
          \item Strecke mit Feldern $-2$ bis $37$,
          \item Bahnsteig A am Feld $-2$ bis $0$,
          \item Bahnsteig B am Feld $14$ bis $16$,
          \item Bahnsteig C am Feld $35$ bis $37$.
        }
      \end{itemize}
    \task
      \IfLanguage{english}{ The train (on field $0$ towards $37$) stands still and has its shift lever at \SI{0}{\kilo\metre\per\hour}.}
      \IfLanguage{ngerman}{Der Zug (auf Feld $0$ in Richtung $37$) steht und hat seinen Schalthebel auf \SI{0}{\kilo\metre\per\hour}.}
      \begin{enumerate}[label=\alph*)]
        \IfLanguage{english}{
          \item If the train accelerates as much as possible, which field can it get to in \emph{nine} rounds?
          \item How many rounds are minimally needed, if the train stops at every station?
        }
        \IfLanguage{ngerman}{
          \item Wenn der Zug maximal beschleunigt, bis zu welchen Feld gelangt er in \emph{neun} Runden?
          \item Wie viele Runden benötigt man minimal, wenn der Zug in jedem Bahnhof halten soll?
        }
      \end{enumerate}
      \IfLanguage{english}{Note the solution steps in a protocol!}
      \IfLanguage{ngerman}{Notiere die Lösungschritte in einem Protokoll!}
      \begin{center}
        \IfLanguage{english}{Example for a protocol:}
        \IfLanguage{ngerman}{Beispiel für ein Protokoll:}
        \\
        %!TEX TS-program = pdflatexmk
%!TEX root = ../handbook.tex

% Copyright 2018 Martin Scheidt (ISC license)
% Permission to use, copy, modify, and/or distribute this file for any purpose with or without fee is hereby granted, provided that the above copyright notice and this permission notice appear in all copies.

\begin{tabular}{cccc|c}
  \toprule
  \IfLanguage{english}{
  Round   & curent                      & (1. Step)   & current position  & (2. Step)                     \\
          & speed                       & Move by     & head of train     & shift lever at                \\

  }
  \IfLanguage{ngerman}{
  Runde   & aktuelle                    & (1. Schritt)& aktuelle          & (2. Schritt)                  \\
          & Geschwindigkeit             & Bewegen um  & Position Zugspitze& Schalthebel auf               \\
  }
  \hline
  \IfLanguage{english}{
  $1$     & \SI{0}{\kilo\metre\per\hour}& $0$ fields  & field $0$         & \SI{40}{\kilo\metre\per\hour} \\
  }
  \IfLanguage{ngerman}{
  $1$     & \SI{0}{\kilo\metre\per\hour}& $0$ Felder  & Feld $0$          & \SI{40}{\kilo\metre\per\hour} \\
  }
  \hline
  $2$     & $\cdots$                    &             &                   &                               \\
  \hline
  $\vdots$&                             &             &                   &                               \\
  \hline 
          &                             &             &                   &                               \\
  \bottomrule
\end{tabular}

      \end{center}
    \task
      \IfLanguage{english}{ The train (on field $0$ towards $37$) just passes through the first station and has its shift lever on maximum speed.}
      \IfLanguage{ngerman}{Der Zug (auf Feld $0$ in Richtung $37$) fährt gerade durch den ersten Bahnhof durch und hat seinen Schalthebel auf der maximalen Geschwindigkeit.}
      \begin{enumerate}[label=a)]
        \IfLanguage{english}{
          \item How many fields does the train need to come to a stop?
          \item How many rounds are needed if the train shall leave the track completely without stopping?
        }
        \IfLanguage{ngerman}{
          \item Wie viele Felder braucht der Zug, bis er zum Stehen gekommen ist?
          \item Wie viele Runden benötigt man, wenn der Zug ohne Halt die Strecke vollständig verlassen soll?
        }
      \end{enumerate}

  \section{\IfLanguage{english}{Sight and Braking Distance}\IfLanguage{ngerman}{Sicht- und Bremsweg}}
    \setup
      \IfLanguage{english}{Unknown line with different visibility conditions:}
      \IfLanguage{ngerman}{Unbekannte Strecke mit verschiedenen Sichtverhältnissen:}
      \\[0.5cm]
      %!TEX TS-program = pdflatexmk
%!TEX root = ../handbook.tex

% Copyright 2018 Martin Scheidt (ISC license)
% Permission to use, copy, modify, and/or distribute this file for any purpose with or without fee is hereby granted, provided that the above copyright notice and this permission notice appear in all copies.

\begin{tabular}{rl}
  \toprule
  \IfLanguage{english}{
    Visibility      & Sight in fields   \\
  }
  \IfLanguage{ngerman}{
    Sichtverhältnis & Sicht in Feldern  \\
  }
  \hline
  \IfLanguage{english}{
    Very good       & 3                 \\
    Normal          & 2                 \\
    Bad             & 1                 \\
  }
  \IfLanguage{ngerman}{
    Sehr gut        & 3                 \\
    Normal          & 2                 \\
    Schlecht        & 1                 \\
  }
  \bottomrule
\end{tabular}

    \task
      \begin{enumerate}[label=\alph*)]
        \IfLanguage{english}{
          \item What is the maximum speed for the train with very good visibility in order to stop in front of an obstacle in time?
          \item How many rounds are minimally needed, to arrive safely with normal visibility in a $14$ fields away station?
          \item How many fields far would you have to be able to see in order to drive \SI{160}{\kilo\metre\per\hour}?
        }
        \IfLanguage{ngerman}{
          \item Wie schnell kann der Zug bei sehr gutem Sichtverhältnis maximal Fahren um vor einem Hindernis rechtzeitig anzuhalten?
          \item Wie viele Runden benötigt man minimal, um gefahrlos bei normalen Sichtverhältnissen in einem $14$ Felder entfernten Bahnhof zu gelangen?
          \item Wie viele Felder weit müsste man sehen können, um \SI{160}{\kilo\metre\per\hour} fahren zu können?
        }
      \end{enumerate}

\chapter{\IfLanguage{english}{Second Stage}\IfLanguage{ngerman}{Zweite Stufe}}
  \section{\IfLanguage{english}{Block Segmentation}\IfLanguage{ngerman}{Blockteilung}}
    \setup
      \IfLanguage{english}{One train and any length of line, with at least 3 complete blocks.
      A block consists of: visual point, distant signal, main signal, signal clearing point and clearing distance.}
      \IfLanguage{ngerman}{Ein Zug und eine beliebig lange Strecke, mit mindestens 3 vollständigen Blöcken.
      Ein Block besteht aus: Sichtpunkt, Vorsignal, Hauptsignal, Signalzugschlussstelle und Räumweg.}
    \task
      \begin{enumerate}[label=\alph*)]
        \IfLanguage{english}{
          \item Place the distant and main signals so that \SI{160}{\kilo\metre\per\hour} can be driven and bad visibility does not lead to impairment!
          \item How many rounds is a block occupied with a train running (complete blocking time)?
        }
        \IfLanguage{ngerman}{
          \item Platziere die Vor- und Hauptsignale mindestens so, dass \SI{160}{\kilo\metre\per\hour} gefahren werden kann und schlechte Sichtverhältnisse nicht zur Beeinträchtigung führt!
          \item Wie viele Runden ist ein Block mit einer Zugfahrt belegt (vollständige Sperrzeit)?
        }
      \end{enumerate}

  \section{\IfLanguage{english}{Traffic Flow}\IfLanguage{ngerman}{Verkehrsfluss}}
    \setup
      \begin{itemize}
        \IfLanguage{english}{
          \item Two different trains with different train dynamics.
          \item Any length of the track, with at least 3 complete blocks.
          \item At the beginning and the end of the track, a station with switches can be arranged or the track can be lead in a circle.
        }
        \IfLanguage{ngerman}{
          \item Zwei verschiedene Züge mit unterschiedlicher Fahrdynamik.
          \item Beliebige Länge der Strecke, mit mindestens 3 vollständigen Blöcken.
          \item Am Anfang und Ende der Strecke kann jeweils ein Bahnhof mit Weichen angeordnet werden oder die Strecke im Kreis geführt werden.
        }
      \end{itemize}
    \task
      \begin{enumerate}[label=\alph*)]
        \IfLanguage{english}{
          \item How many rounds are needed if both trains shall run unimpeded and the fast train runs in front of the slow train?
          \item How many rounds are needed if both trains shall run unimpeded and the slow train runs in front of the fast train?
        }
        \IfLanguage{ngerman}{
          \item Wie viele Runden benötigt man, wenn beide Züge behinderungsfrei fahren sollen und der \emph{schnelle} vor dem \emph{langsamen} Zug fährt?
          \item Wie viele Runden benötigt man, wenn beide Züge behinderungsfrei fahren sollen und der \emph{langsame} vor dem \emph{schnellen} Zug fährt?
        }
      \end{enumerate}