%!tex root=handbook.tex
% ===============================
%    Kodierung und Sprache
% ===============================
% \usepackage[T1]{fontenc} % ermoeglicht die Silbentrennung von Woertern mit Umlauten
\usepackage{ucs}
\usepackage[utf8x]{inputenc}
\usepackage[ngerman]{babel}
\usepackage[ngerman]{translator}

% -----[ revision history ]----------
\usepackage{vhistory}
% -----[ Creative Commons License ]----------
\usepackage[scale=1.5]{ccicons}

% ===============================
%    Tabellen und Abbildungen
% ===============================
\usepackage{graphicx} % ermoeglicht einbinden von Graphiken
\usepackage{subfigure} % ermoeglicht Bilder nebeneinander aufzureihen
\usepackage{tabularx} % ermoeglicht seitenbreite Tabellen
\usepackage{booktabs} % schoenere Tabellen
\usepackage{multirow}
%\usepackage{longtable} % ermoeglicht Tabellen mit Seitenumbruch
\usepackage{amsmath}
\usepackage{xfrac} % ermoeglich schräg gestellte Brüche mit \sfrac{}{}
\usepackage[locale=DE]{siunitx} % for SI-Units
\sisetup{
  per-mode=fraction,
  fraction-function=\sfrac
}
\usepackage{wasysym} % \permil
\usepackage{enumerate} % Nummeriung aendern
\usepackage{listings}
\usepackage{rotating}
\usepackage{tikz}
% \usepackage{framed} % Rahmen zeichnen

% ===============================
%    chapter/section numbering
% ===============================
\newcounter{question}
\newcommand{\aufgabe}{\stepcounter{question}\bigskip\par\noindent{\normalfont\large\bfseries Aufgabe \arabic{question}}\medskip\par\noindent}
\makeatletter\@addtoreset{chapter}{part}\makeatother%
