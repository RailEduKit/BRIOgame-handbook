%!TEX TS-program = pdflatexmk

% Copyright 2019 Martin Scheidt (Attribution 4.0 International, CC-BY 4.0)
% You are free to copy and redistribute the material in any medium or format. You are free to remix, transform, and build upon the material for any purpose, even commercially. You must give appropriate credit, provide a link to the license, and indicate if changes were made. You may not apply legal terms or technological measures that legally restrict others from doing anything the license permits. No warranties are given.

\documentclass{beamer}
\usepackage[
  size=a4,
]{beamerposter}
\beamertemplatenavigationsymbolsempty

\def\ROOT{./..}

%!TEX TS-program = pdflatexmk
%!TEX root = handbook.tex

% Copyright 2019 Martin Scheidt (Attribution 4.0 International, CC-BY 4.0)
% You are free to copy and redistribute the material in any medium or format. You are free to remix, transform, and build upon the material for any purpose, even commercially. You must give appropriate credit, provide a link to the license, and indicate if changes were made. You may not apply legal terms or technological measures that legally restrict others from doing anything the license permits. No warranties are given.

% --------[  Coding and Language  ]----------
\usepackage[utf8x]{inputenc}
\usepackage[T1]{fontenc}
\usepackage[main=english,ngerman]{babel}
\usepackage{iflang}
\newcommand{\IfLanguage}[2]{\IfLanguageName{#1}{#2}{}}

% % --------[ Creative Commons License ]-------
\usepackage[scale=1.5]{ccicons}


% --------[ Layout  ]-----------
\usepackage{lmodern,microtype,mathptmx,courier}
\usepackage[scaled=0.92]{helvet}

% -------[ Symbols ]---------
\usepackage{amsmath,amsthm}
\usepackage{xfrac} % provides slanted fractures with \sfrac{}{}
\usepackage{wasysym} % \permil
\usepackage{siunitx} % for SI-Units
\sisetup{
  per-mode=fraction,
  fraction-function=\sfrac
}

\usepackage{tikz,adjustbox}


%!TEX TS-program = pdflatexmk
%!TEX root = handbook.tex

% Copyright 2019 Martin Scheidt (Attribution 4.0 International, CC-BY 4.0)
% You are free to copy and redistribute the material in any medium or format. You are free to remix, transform, and build upon the material for any purpose, even commercially. You must give appropriate credit, provide a link to the license, and indicate if changes were made. You may not apply legal terms or technological measures that legally restrict others from doing anything the license permits. No warranties are given.

\usepackage[prefix=]{xcolor-solarized}
\definecolor{signalgreen}{RGB}{0,181,26}
\definecolor{signalyellow}{RGB}{255,230,0}
\definecolor{signalred}{RGB}{255,0,0}

%!TEX TS-program = pdflatexmk

% Copyright 2018 RailToolKit (Attribution 4.0 International, CC-BY 4.0)
% You are free to copy and redistribute the material in any medium or format. You are free to remix, transform, and build upon the material for any purpose, even commercially. You must give appropriate credit, provide a link to the license, and indicate if changes were made. You may not apply legal terms or technological measures that legally restrict others from doing anything the license permits. No warranties are given.

\tikzset{
  stop/.pic={
    \draw[fill=signalred] (0,0) circle (0.3);
    \draw[double] (-0.3,0) -- (0.3,0);
  };
}
\tikzset{
  approach/.pic={
    \draw[fill=signalyellow] (0,0) circle (0.3);
    \draw[double] (-0.22,-0.22) -- ++(0.44,0.44);
  };
}
\tikzset{
  clear/.pic={
    \draw[fill=signalgreen] (0,0) circle (0.3);
    \draw[double] (0,-0.3) -- (0,0.3);
  };
}
\tikzset{
  end_of_train/.pic={
    \fill[signalred] (-0.7,-0.5) -- (0,0) -- (-0.7,0.5) -- cycle;
    \fill[signalred] ( 0.7,-0.5) -- (0,0) -- ( 0.7,0.5) -- cycle;
    \draw (-0.7,-0.5) rectangle (0.7,0.5);
  };
}

\newlength{\trainlength}\setlength{\trainlength}{20cm}
\newlength{\trainheight}\setlength{\trainheight}{3cm}
\newlength{\trainwidth}\setlength{\trainwidth}{4cm}

\tikzset{
  lever_positions/.pic={
    \draw[->,>=latex,line width=1.5pt,green] ( 1.5,0)  .. controls ( 2,-0.5) and ( 3,-0.5) .. ( 3.5,0);
    \draw[->,>=latex,line width=1.5pt,green] ( 4.5,0)  .. controls ( 5,-0.5) and ( 6,-0.5) .. ( 6.5,0);
    \draw[->,>=latex,line width=1.5pt,green] ( 7.5,0)  .. controls ( 8,-0.5) and ( 9,-0.5) .. ( 9.5,0);
    \draw[->,>=latex,line width=1.5pt,green] (10.5,0)  .. controls (11,-0.5) and (12,-0.5) .. (12.5,0);
    \draw[->,>=latex,line width=1.5pt,green] (13.5,0)  .. controls (14,-0.5) and (15,-0.5) .. (15.5,0);
    %
    \draw[<-,>=latex,line width=1.5pt,green] ( 1.5,2)  .. controls ( 2,2.5) and ( 3,2.5) .. ( 3.5,2);
    \draw[<-,>=latex,line width=1.5pt,green] ( 4.5,2)  .. controls ( 5,2.5) and ( 6,2.5) .. ( 6.5,2);
    \draw[<-,>=latex,line width=1.5pt,green] ( 7.5,2)  .. controls ( 8,2.5) and ( 9,2.5) .. ( 9.5,2);
    \draw[<-,>=latex,line width=1.5pt,green] (10.5,2)  .. controls (11,2.5) and (12,2.5) .. (12.5,2);
    \draw[<-,>=latex,line width=1.5pt,green] (13.5,2)  .. controls (14,2.5) and (15,2.5) .. (15.5,2);
    % lever positiions
    \foreach \x in {0,3,6,9,12,15}
      \draw [fill=white] (\x,0) rectangle ++(2,2);
    % labels
    \node[align=center] at ( 1,1.6) {\si{\kilo\metre\per\hour}};
    \node[align=center] at ( 1,1  ) {\Huge $0$};
    \node[align=center] at ( 1,0.3) {\Large \color{blue} $0$
      \IfLanguage{english}{fields}
      \IfLanguage{ngerman}{Felder}
    };
    \node[align=center] at ( 4,1) {\Large \color{blue}
      \IfLanguage{english}{\Large \color{blue} suspend\\\Large \color{blue} a\\\Large \color{blue} round}
      \IfLanguage{ngerman}{\Large \color{blue} eine\\\Large \color{blue} Runde\\\Large \color{blue} aussetzen}
    };
    \node[align=center] at ( 7,1.6) {\si{\kilo\metre\per\hour}};
    \node[align=center] at ( 7,1  ) {\Huge $40$};
    \node[align=center] at ( 7,0.3) {\Large \color{blue} $1$
      \IfLanguage{english}{fields}
      \IfLanguage{ngerman}{Felder}
    };
    \node[align=center] at (10,1) {\Large \color{blue}
      \IfLanguage{english}{\Large \color{blue} suspend\\\Large \color{blue} a\\\Large \color{blue} round}
      \IfLanguage{ngerman}{\Large \color{blue} eine\\\Large \color{blue} Runde\\\Large \color{blue} aussetzen}
    };
    \node[align=center] at (13,1.6) {\si{\kilo\metre\per\hour}};
    \node[align=center] at (13,1  ) {\Huge $80$};
    \node[align=center] at (13,0.3) {\Large \color{blue} $2$
      \IfLanguage{english}{fields}
      \IfLanguage{ngerman}{Felder}
    };
    \node[align=center] at (16,1.6) {\si{\kilo\metre\per\hour}};
    \node[align=center] at (16,1  ) {\Huge $120$};
    \node[align=center] at (16,0.3) {\Large \color{blue} $3$
      \IfLanguage{english}{fields}
      \IfLanguage{ngerman}{Felder}
    };
  };
}

\begin{document}
  \selectlanguage{ngerman} % currently supported: english, ngerman
  \tikzset{every path/.style={ultra thick}}
  \begin{frame}
    \begin{tikzpicture}[font=\sffamily]
      \coordinate (base) at (0,0);
      % coordinates
      \path ([shift={(base)}] 0,0) coordinate (A3)
            -- ++(0,-\trainheight) coordinate (A4)
            -- ++(\trainlength, 0) coordinate (A1)
            -- ++(0, \trainheight) coordinate (A2);
      \path (A2) -- ++ ( 1,-1) coordinate (A5);
      \path (A1) -- ++ ( 1, 1) coordinate (A6);
      \path ([shift={(base)}] 0,0) coordinate (B3)
            -- ++(0, \trainwidth)  coordinate (B4)
            -- ++(-\trainheight,0) coordinate (B7)
            -- ++(0,-\trainwidth)  coordinate (B8);
      \path (B4) -- ++ (-1, 1) coordinate (B5);
      \path (B7) -- ++ ( 1, 1) coordinate (B6);
      \path (B8) -- ++ ( 1,-1) coordinate (B1);
      \path (B3) -- ++ (-1,-1) coordinate (B2);
      \path ([shift={(base)}] 0,0) coordinate (C3)
            -- ++(0, \trainwidth)  coordinate (C4)
            -- ++(\trainlength, 0) coordinate (C1)
            -- ++( 0.375\trainwidth,-0.5\trainwidth) coordinate (C5)
            -- ++(-0.375\trainwidth,-0.5\trainwidth) coordinate (C2);
      \path (C4)                    coordinate (D4)
            -- ++(0, \trainheight)  coordinate (D3)
            -- ++(\trainlength, 0)  coordinate (D2)
            -- ++(0,-\trainheight)  coordinate (D1);
      \path (D1)                    coordinate (E1)
            -- ++(0, \trainheight)  coordinate (E2)
            -- ++(0.625\trainwidth, 0) coordinate (E3)
            -- ++(0,-\trainheight)  coordinate (E4)
            -- ++(-1,-1)       coordinate (E5);
      \path (E1) -- ++ ( 1,-1) coordinate (E6);
      \path (E4)                    coordinate (F4)
            -- ++(0, \trainheight)  coordinate (F3)
            -- ++(0.625\trainwidth, 0) coordinate (F2)
            -- ++(0,-\trainheight)  coordinate (F1)
            -- ++(-1,-1)       coordinate (F6);
      \path (F4) -- ++ ( 1,-1) coordinate (F5);
      % drawing
      \draw (A2) rectangle (A4);
      \draw (B4) rectangle (B8);
      \draw (C2) -- (C3) -- (C4) -- (C1) -- (C5) -- cycle;
      \draw (D1) rectangle (D3);
      \draw (E2) rectangle (E4);
      \draw (F2) rectangle (F4);
      % labels
      \pic[scale=0.75] at (0.1\trainlength,0.35\trainwidth) {lever_positions};
      \draw ([shift={(base)}] 1,0)
            -- (0,0.5\trainwidth)
            -- (1,\trainwidth)
            -- cycle;
      \node[rotate=90] at (0.03\trainlength,0.5\trainwidth) {
        \IfLanguage{english}{\Large Coupling}
        \IfLanguage{ngerman}{\Large Kupplung}
      };
      % adhesive edge
      \tikzset{every path/.style={thin,base1}}
      \draw (A2) -- (A5) -- (A6) -- (A1);
      \draw (B4) -- (B5) -- (B6) -- (B7);
      \draw (B8) -- (B1) -- (B2) -- (B3);
      \draw (E1) -- (E6) -- (E5) -- (E4);
      \draw (F1) -- (F6) -- (F5) -- (F4);
      %
    \end{tikzpicture}
    %!TEX TS-program = pdflatexmk
%!TEX root = ../handbook.tex

% Copyright 2021 Martin Scheidt (Attribution 4.0 International, CC-BY 4.0)
% You are free to copy and redistribute the material in any medium or format. You are free to remix, transform, and build upon the material for any purpose, even commercially. You must give appropriate credit, provide a link to the license, and indicate if changes were made. You may not apply legal terms or technological measures that legally restrict others from doing anything the license permits. No warranties are given.

\vspace*{\fill}

\IfLanguage{english}{\noindent\large Version \vhCurrentVersion\ from \vhCurrentDate}
\IfLanguage{ngerman}{\noindent\large Version \vhCurrentVersion\ vom \vhCurrentDate}\\[0.3cm]

\IfLanguage{english}{
  \ccLogo \ccAttribution ~This work is licensed under a Creative Commons License (CC BY 4.0).
}
\IfLanguage{ngerman}{
  \ccLogo \ccAttribution ~Dieses Werk steht unter der Creative Commons Lizens (CC BY 4.0).
}

  \end{frame}
  \begin{frame}
    \begin{tikzpicture}[font=\sffamily]
      \coordinate (base) at (0,0);
      % coordinates
      \path ([shift={(base)}] 0,0) coordinate (A3)
            -- ++(0,-\trainheight) coordinate (A4)
            -- ++(\trainlength, 0) coordinate (A1)
            -- ++(0, \trainheight) coordinate (A2);
      \path ([shift={(base)}] 0,0) coordinate (B3)
            -- ++(0, \trainwidth)  coordinate (B4)
            -- ++(-\trainheight,0) coordinate (B7)
            -- ++(0,-\trainwidth)  coordinate (B8);
      \path (B4) -- ++ (-1, 1) coordinate (B5);
      \path (B7) -- ++ ( 1, 1) coordinate (B6);
      \path (B8) -- ++ ( 1,-1) coordinate (B1);
      \path (B3) -- ++ (-1,-1) coordinate (B2);
      \path ([shift={(base)}] 0,0) coordinate (C3)
            -- ++(0, \trainwidth)  coordinate (C4)
            -- ++(\trainlength, 0) coordinate (C1)
            -- ++(0,-\trainwidth)  coordinate (C2);
      \path (C2) coordinate (D2)
            -- ++(0, \trainwidth)  coordinate (D1)
            -- ++( \trainheight,0) coordinate (D6)
            -- ++(0,-\trainwidth)  coordinate (D5);
      \path (D6) -- ++ (-1, 1) coordinate (D7);
      \path (D1) -- ++ ( 1, 1) coordinate (D8);
      \path (D2) -- ++ ( 1,-1) coordinate (D3);
      \path (D5) -- ++ (-1,-1) coordinate (D4);
      \path (C4)                    coordinate (E4)
            -- ++(0, \trainheight)  coordinate (E3)
            -- ++(\trainlength, 0)  coordinate (E2)
            -- ++(0,-\trainheight)  coordinate (E1);
      % drawing
      \draw (A2) rectangle (A4);
      \draw (B4) rectangle (B8);
      \draw (C2) rectangle (C4);
      \draw (D1) rectangle (D5);
      \draw (E2) rectangle (E4);
      % labels
      \pic at (-0.5\trainheight,0.5\trainwidth) {end_of_train};
      \draw ([shift={(base)}] \trainlength,0.5\trainwidth)
            -- ++(-1,0.5\trainwidth)
            -- ++(0,-\trainwidth)
            -- cycle;
      \node[rotate=-90] at (0.97\trainlength,0.5\trainwidth) {
        \IfLanguage{english}{\Large Coupling}
        \IfLanguage{ngerman}{\Large Kupplung}
      };
      % adhesive edge
      \tikzset{every path/.style={thin,base1}}
      \draw (B4) -- (B5) -- (B6) -- (B7);
      \draw (B8) -- (B1) -- (B2) -- (B3);
      \draw (D2) -- (D3) -- (D4) -- (D5);
      \draw (D1) -- (D8) -- (D7) -- (D6);
      %
    \end{tikzpicture}
  \end{frame}
  \begin{frame}
    \IfLanguage{english}{\Huge \textcolor{blue}{Control Lever:}}
    \IfLanguage{ngerman}{\Huge \textcolor{blue}{Schalthebelpositionen:}}\\
    \begin{tikzpicture}[font=\sffamily]
      \begin{scope}[scale=1.5,transform shape]
        \pic at (0,0) {lever_positions};
     \end{scope}
    \end{tikzpicture}
    \vfill
    %!TEX TS-program = pdflatexmk

% Copyright 2021 Martin Scheidt (Attribution 4.0 International, CC-BY 4.0)
% You are free to copy and redistribute the material in any medium or format. You are free to remix, transform, and build upon the material for any purpose, even commercially. You must give appropriate credit, provide a link to the license, and indicate if changes were made. You may not apply legal terms or technological measures that legally restrict others from doing anything the license permits. No warranties are given.

% \IfLanguage{english}{Sequence per round:
% \begin{enumerate}
%   \item calling of routes (optional)
%   \item Set signals to CLEAR (optional)
%   \item Select control lever position (optional)
%   \item \emph{move all trains according to the control lever position!}
%   \item \emph{Execute stop case for signals!}
%   \item release of routes (optional)
% \end{enumerate}
% }
% \IfLanguage{ngerman}{Ablauf pro Runde:
\begin{enumerate}
  \item Fahrstraße sichern
    \begin{enumerate}[a)]
      \item Weichen im befahrenen Teil der Fahrstraße einstellen und sichern
      \item Flankenschutz herstellen
      \item Durchrutschweg sichern
    \end{enumerate}
  \item Signale auf Fahrt stellen
  \item Schalthebelposition ändern (Beschleunigen/Bremsen)
  \item Züge bewegen
  \item Haltfall von Signalen ausführen
  \item Fahrstraßen auflösen
\end{enumerate}
% }
  \end{frame}
\end{document}