%!TEX TS-program = pdflatexmk
%!TEX root = ../handbook.tex

% Copyright 2021 Martin Scheidt (Attribution 4.0 International, CC-BY 4.0)
% You are free to copy and redistribute the material in any medium or format. You are free to remix, transform, and build upon the material for any purpose, even commercially. You must give appropriate credit, provide a link to the license, and indicate if changes were made. You may not apply legal terms or technological measures that legally restrict others from doing anything the license permits. No warranties are given.

{\bigskip\par\noindent{\huge\bfseries\sffamily\IfLanguage{english}{Aim}\IfLanguage{ngerman}{Ziel}}\medskip\par\noindent}

\noindent
\IfLanguage{english}{The aim of the learning game is to simulate and experience the driving dynamics of trains in the context of block division.
Real continuous dimensions time ($t$) and distance ($s$) are assigned to discrete units of laps ($t$) and spaces ($s$).
Thus, the simulation is round-based in order to imitate the steps of a computer.}

\IfLanguage{ngerman}{Ziel des Lernspieles ist die Fahrdynamik von Zügen im Zusammenhang mit Blockteilung zu simulieren und zu erfahren.
Reale kontinuierliche Größen Zeit ($t$) und Strecke ($s$) werden dabei in diskrete Einheiten von Runden ($t$) und Felder ($s$) eingeteilt.
Die Simulation erfolgt also Rundenbasiert, um im Schrittverfahren einen Computer nachzuahmen.}
