%!TEX TS-program = pdflatexmk
%!TEX root = ../handbook.tex

% Copyright 2020 Martin Scheidt (Attribution 4.0 International, CC-BY 4.0)
% You are free to copy and redistribute the material in any medium or format. You are free to remix, transform, and build upon the material for any purpose, even commercially. You must give appropriate credit, provide a link to the license, and indicate if changes were made. You may not apply legal terms or technological measures that legally restrict others from doing anything the license permits. No warranties are given.

\section{Zwieschutzweichen Aufgabe}

  \roles
    \begin{itemize}
      \item Spielleiter
      \item drei Triebfahrzeugführer
      \item Fahrdienstleiter
    \end{itemize}

  \material
    \begin{itemize}
      \item Gleise
      \item Weichen
      \item Signale und Zugschlusstellen
      \item drei beliebige Züge mit Fahrbrett, Zugschluss- und -spitzensignal
    \end{itemize}

  \setup
    \tikzfigure{challenge3_setup3.tikz}
    Die Züge 1 (Nahverkehrszug) und 2 (Güterzug) stehen im Bahnhof und sollen auf das rechte Streckengleis ausfahren. Zug 3 (Fernverkehrszug) fährt mit maximaler Geschwindigkeit und soll über das freie Gleis weiterfahren.

  \task
    \begin{enumerate}[label=\alph*)]
      \item Ergänzt die Infrastruktur mit Vorsignalen, Signalzugschlussstellen und Fahrstraßenzugschlussstellen!
      \item Sichert die Ausfahrstraße für Zug 1!
      \item Sichert die Einfahrstraße für Zug 3! Welches Problem tritt auf? Wie kann es gelöst werden?
      \item Spielt einige Runden, bis Zug 1 den Spielbereich verlassen hat. Sichert dann die Ausfahrstraße für Zug 2. Welches Problem tritt auf? Wie kann es gelöst werden?
    \end{enumerate}