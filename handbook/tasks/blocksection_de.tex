%!TEX TS-program = pdflatexmk
%!TEX root = ../handbook.tex

% Copyright 2018 - 2022 Martin Scheidt (Attribution 4.0 International, CC-BY 4.0)
% You are free to copy and redistribute the material in any medium or format. You are free to remix, transform, and build upon the material for any purpose, even commercially. You must give appropriate credit, provide a link to the license, and indicate if changes were made. You may not apply legal terms or technological measures that legally restrict others from doing anything the license permits. No warranties are given.

\section{Blockteilung}

  \roles
    \begin{itemize}
      \item Spielleiter
      \item Infrastrukturplaner
    \end{itemize}

  \material
    \begin{itemize}
      \item Gleise
      \item drei vollständige Blöcke mit Vorsignal, Hauptsignal und Signalzugschlusstelle
      \item beliebiger Zug mit zugehörigen Fahrdynamikmodell, Zugschluss- und -spitzensignal
    \end{itemize}

  \setup
    Es soll eine Strecke geplant werden, auf der mit Blocklogik mehrere Züge verkehren können.

  \task
    \begin{enumerate}[label=\alph*)]
      \item Platziert die Vorsignale, Hauptsignale und Signalzugschlussstellen so, dass verschiedene Züge verkehren können und ohne dass schlechte Sichtverhältnisse zu Beeinträchtigungen führen.
      \item Was ist hierfür der minimale und was ist der maximale Blockabstand?
      \item Was passiert, wenn Ihr den minimalen Blockabstand unterschreitet?
      \item Wie viele Runden dauert die vollständige Sperrzeit für die Fahrt durch einen beliebigen Blockabschnitt (Sichtzeit, Annäherungszeit, Fahrzeit im Block, Räumfahrzeit)?
    \end{enumerate}
