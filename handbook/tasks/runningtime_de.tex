%!TEX TS-program = pdflatexmk
%!TEX root = ../handbook.tex

% Copyright 2021 Martin Scheidt (Attribution 4.0 International, CC-BY 4.0)
% You are free to copy and redistribute the material in any medium or format. You are free to remix, transform, and build upon the material for any purpose, even commercially. You must give appropriate credit, provide a link to the license, and indicate if changes were made. You may not apply legal terms or technological measures that legally restrict others from doing anything the license permits. No warranties are given.

\section{Fahrzeitberechnung Aufgaben}

  \roles
    \begin{itemize}
      \item Spielleiter
      \item Triebfahrzeugführer des Nahverkehrszuges
    \end{itemize}

  \material
    \begin{itemize}
      \item Gleise
      \item drei Bahnsteige
      \item Nahverkehrszug (braun) mit passendem Fahrdynamikmodell, Zugschluss- und -spitzensignal
    \end{itemize}

  \setup
    \tikzfigure{challenge1_setup.tikz}

  \task
    Der Zug steht auf Feld 0 und sein Schalthebel steht auf \SI{0}{\kilo\metre\per\hour} (0 Felder).
    \begin{enumerate}[label=\alph*)]
      \item Wenn der Zug maximal beschleunigt, bis zu welchem Feld gelangt er in \emph{neun} Runden?
      \item Wie viele Runden benötigt der Zug minimal, wenn er an jedem Bahnsteig für eine Runde anhält?
    \end{enumerate}

  \setup
    Der Zug befindet sich auf Feld 0 und sein Schalthebel steht auf Höchstgeschwindigkeit
    \begin{enumerate}[label=\alph*),resume]
      \item Wie viele Runden benötigt man, wenn der Zug ohne Halt die Strecke vollständig verlassen soll?
    \end{enumerate}
