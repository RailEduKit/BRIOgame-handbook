%!TEX TS-program = pdflatexmk
%!TEX root = ../handbook.tex

% Copyright 2018 - 2022 Martin Scheidt (Attribution 4.0 International, CC-BY 4.0)
% You are free to copy and redistribute the material in any medium or format. You are free to remix, transform, and build upon the material for any purpose, even commercially. You must give appropriate credit, provide a link to the license, and indicate if changes were made. You may not apply legal terms or technological measures that legally restrict others from doing anything the license permits. No warranties are given.

\section{Fahrzeitberechnung Aufgaben}

  \roles
    \begin{itemize}
      \item Spielleiter
      \item Triebfahrzeugführer des Nahverkehrszuges
    \end{itemize}

  \material
    \begin{itemize}
      \item Gleise
      \item drei Bahnsteige
      \item Nahverkehrszug (braun) mit passendem Fahrdynamikmodell, Zugschluss- und -spitzensignal
    \end{itemize}

  \setup
    \tikzfigure{challenge1_setup.tikz}

  \task
    Der Zug steht auf Feld 0 und sein Schalthebel steht auf \SI{0}{\kilo\metre\per\hour} (0 Felder).
    \begin{enumerate}[label=\alph*)]
      \item Der Zug beschleunigt jetzt maximal, bis zu welchem Feld ist er in \textbf{neun} Runden gelangt?
      \item Der Zug startet wieder bei Feld 0 und soll jetzt an beiden Bahnhöfen für jeweils 1 Runde halten (nachdem der Schalthebel auf \SI{0}{\kilo\metre\per\hour} steht), damit Fahrgäste umsteigen können. Wie viele Runden braucht er \textbf{mindestens}? 
    \end{enumerate}

  \setup
    Der Zug befindet sich auf Feld 0 und sein Schalthebel steht auf Höchstgeschwindigkeit
    \begin{enumerate}[label=\alph*),resume]
      \item Der Zug fährt ohne Halt und soll die Strecke vollständig verlassen. Wie viele Runden benötigt er?
    \end{enumerate}
