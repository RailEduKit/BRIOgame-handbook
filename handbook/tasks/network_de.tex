%!TEX TS-program = pdflatexmk
%!TEX root = ../handbook.tex

% Copyright 2018 - 2022 Martin Scheidt (Attribution 4.0 International, CC-BY 4.0)
% You are free to copy and redistribute the material in any medium or format. You are free to remix, transform, and build upon the material for any purpose, even commercially. You must give appropriate credit, provide a link to the license, and indicate if changes were made. You may not apply legal terms or technological measures that legally restrict others from doing anything the license permits. No warranties are given.

  \roles
    \begin{itemize}
      \item Spielleiter (1-3)
      \item Disponent (1-3)
      \item Fahrdienstleiter (ca. 2-5)
      \item Triebfahrzeugführer (7)
    \end{itemize}

  Der Disponent (oder das entsprechende Team) geben Anweisungen an die Fahrdienstleiter in welcher Reihenfolge und über welche Gleise die Züge verkehren sollen.

  \material
    \begin{itemize}
      \item fünf vollständige Sets
      \item Züge entsprechend dem Fahrplankonzept
    \end{itemize}

  \setup
    Es ist ein Gebiet mit Verkehrsstationen und dem dargestellten Fahrplankonzept gegeben. Jede Linie entspricht einem Zugpaar.\\[0.5cm]
    \tikzfigure{challenge4_setup.tikz}
    \begin{itemize}
      \item Ein Unterwegshalt muss an einem Bahnsteig der entsprechenden Länge stattfinden
      \item Ein Wendehalt muss an einem Bahnsteig der entsprechenden Länge stattfinden.
      \item Das Ein- und Ausbrechen der Züge geschieht am Gleisende über Auffahrrampen.
      \item Das Wenden mit Abräumen entspricht einer Fahrt in und aus dem Anschluss (Rampe am Gleisende).
    \end{itemize}

  \task 
    \emph{Infrastrukturgestaltung}: Zunächst ist die gesamte Eisenbahninfrastruktur zu entwerfen und zu errichten:
    \begin{enumerate}[label=\alph*)]
      \item Die Verkehrsstationen können beliebige Betriebsstellen sein, welche die Ausführung der entsprechenden Funktion (Unterwegshalt, Wendehalt oder Anschlussbedienung) ermöglichen.
      \item Die Strecken zwischen den Verkehrsstationen können ein- oder zweigleisig sein.
      \item Streckenverzweigungen sind entweder in den Knotenbahnhöfen oder auf der freien Strecke in Form von Abzweigstellen möglich.
      \item Der Abstand der Verkehrsstationen ist nicht konkret vorgegeben. Er sollte jedoch immer mindestens einem ganzen Blockabschnitt entsprechen.
    \end{enumerate}

    Ziel ist es jedes Zugpaar einmal fahren zu lassen, hierbei möglichst wenige Runden zu benötigen und gleichfalls sparsam mit den begrenzten Infrastrukturelementen (insbesondere Weichen, Signale und mehrgleisige Abschnitte) umzugehen.

    Weitere Randbedingungen:
    \begin{itemize}
      \item Weichen können mit max. 80 km/h im abzweigenden Strang befahren werden.
      \item Vorsignale müssen jeweils im Bremswegabstand der verkehrenden Züge gesetzt werden.
    \end{itemize}

  \task
    \emph{Startpunkt der Zugpaare festlegen}: Legt gemeinsam für jedes Zugpaar einen Startpunkt fest. Diese kann entweder eine Verkehrsstation mit einer Wende oder eine der Einbruchsstellen sein. Stellt die Züge hierzu bei einer Einbruchsstelle in der vorgesehenen Reihenfolge vor die entsprechende Rampe. Bei Einem Start an einer Wendestelle werden Personenzüge an den entsprechenden Bahnsteig und Güterzüge vor die Rampe der Anschlussstelle gestellt.


  \task
    \emph{Fahrbetrieb}: Beginnt nun damit den Fahrbetrieb durchzuführen. Die Regeln hierzu habt Ihr in den vorangegangenen Tutorials und Aufgaben gelernt. Jeder soll hierbei primär seine Rolle ausführen. Es ist aber natürlich erlaubt, sich gegenseitig zu beraten. Achtet nur darauf den Runden- und Phasenablauf insgesamt einzuhalten. Hierfür sind vor allem die Spielleiter verantwortlich.

