%!TEX TS-program = pdflatexmk
%!TEX root = ../handbook.tex

% Copyright 2018 - 2022 Martin Scheidt (Attribution 4.0 International, CC-BY 4.0)
% You are free to copy and redistribute the material in any medium or format. You are free to remix, transform, and build upon the material for any purpose, even commercially. You must give appropriate credit, provide a link to the license, and indicate if changes were made. You may not apply legal terms or technological measures that legally restrict others from doing anything the license permits. No warranties are given.

\section{Durchrutschweg Aufgaben}

  \roles
    \begin{itemize}
      \item Spielleiter
      \item Triebfahrzeugführer eines beliebigen Zuges
      \item Triebfahrzeugführer eines anderen beliebigen Zuges
      \item Fahrdienstleiter
    \end{itemize}

  \material
    \begin{itemize}
      \item Gleise
      \item Weichen
      \item Signale und Zugschlusstellen
      \item zwei beliebige Züge mit zugehörigen Fahrdynamikmodell, Zugschluss- und -spitzensignal
    \end{itemize}

  \setup
    \tikzfigure{challenge3_setup2.tikz}
    Beide Züge fahren aktuell mit maximaler Geschwindigkeit. Zug 1 soll im Bahnhof möglichst durchfahren. Zug 2 hat einen Verkehrshalt von 2 Runden Dauer im Bahnhof. Die Weichen dürfen im abzweigenden Strang mit \SI{80}{\kilo\metre\per\hour} befahren werden. Der Durchrutschweg beträgt 2 Felder.

  \task
    \begin{enumerate}[label=\alph*)]
      \item Ergänzt die Infrastruktur mit Vorsignalen, Signalzugschlussstellen und Fahrstraßenzugschlussstellen!
      \item Entscheidet begründet welcher Zug auf welches Gleis fahren soll. Welche Probleme können auftreten?
      \item Nach wie viel Runden ist Zug 2 im Bahnhof zum Stehen gekommen?
      \item Nach wie viel Runden hat Zug 1 den Bahnhof vollständig verlassen?
      \item Fertigt vom Bahnhof einen Verschlussplan an!
    \end{enumerate}
    