%!TEX TS-program = pdflatexmk
%!TEX root = ../handbook.tex

% Copyright 2018 - 2022 Martin Scheidt (Attribution 4.0 International, CC-BY 4.0)
% You are free to copy and redistribute the material in any medium or format. You are free to remix, transform, and build upon the material for any purpose, even commercially. You must give appropriate credit, provide a link to the license, and indicate if changes were made. You may not apply legal terms or technological measures that legally restrict others from doing anything the license permits. No warranties are given.

\section{Sicht- und Bremsweg Aufgaben}

  \roles
    \begin{itemize}
      \item Spielleiter
      \item Triebfahrzeugführer des Nahverkehrszuges
      \item Saboteur
    \end{itemize}
    Sonderregeln Saboteur:\\
    Der Saboteur schreibt sich je Durchgang das Feld (zwischen Feld 5 und 15) verdeckt auf einen Zettel, auf das er sein Hindernis platzieren möchte. Erst wenn das notierte Feld vom Zug aus sichtbar ist, legt er das Hindernis auf das entsprechende Feld. Ziel des Saboteurs ist es einen Unfall zu provozieren. Wenn der Zug vor dem Hindernis anhält, kann es innerhalb einer Runde entfernt werden.


  \material
    \begin{itemize}
      \item Gleise
      \item ein beliebiges Hindernis
      \item Nahverkehrszug (braun) mit passendem Fahrdynamikmodell, Zugschluss- und -spitzensignal
    \end{itemize}

  \setup
    Es liegt eine unbekannte Strecke mit verschiedenen Sichtverhältnissen vor Euch:\\
    %!TEX TS-program = pdflatexmk
%!TEX root = ../handbook.tex

% Copyright 2018 Martin Scheidt (ISC license)
% Permission to use, copy, modify, and/or distribute this file for any purpose with or without fee is hereby granted, provided that the above copyright notice and this permission notice appear in all copies.

\begin{tabular}{rl}
  \toprule
  \IfLanguage{english}{
    Visibility      & Sight in fields   \\
  }
  \IfLanguage{ngerman}{
    Sichtverhältnis & Sicht in Feldern  \\
  }
  \hline
  \IfLanguage{english}{
    Very good       & 3                 \\
    Normal          & 2                 \\
    Bad             & 1                 \\
  }
  \IfLanguage{ngerman}{
    Sehr gut        & 3                 \\
    Normal          & 2                 \\
    Schlecht        & 1                 \\
  }
  \bottomrule
\end{tabular}
\\ \\
Der Nahverkehrszug steht auf Feld 0 und soll möglichst schnell Feld 20 erreichen.

\newpage

  \task
    \begin{enumerate}[label=\alph*)]
      \item Fahrt einen Durchgang bei sehr guten, einen Durchgang bei normalen und einen Durchgang bei schlechten Sichtverhältnissen. Ihr könnt Eure Rollen zwischen den Runden tauschen. Wie oft habt Ihr einen Unfall? Falls Ihr ankommt, wie viele Runden habt Ihr gebraucht?
      \item Probiert aus, wie schnell der Zug bei den unterschiedlichen Sichtverhältnissen jeweils fahren darf, um sicher vor einem Hindernis anhalten zu können?
      \item Wie viele Felder weit müsste man sehen können, um bei \SI{160}{\kilo\metre\per\hour} (4 Felder) vor jedem Hindernis sicher anhalten zu können?
    \end{enumerate}
