%!TEX TS-program = pdflatexmk
%!TEX root = ../handbook.tex

% Copyright 2020 Martin Scheidt (Attribution 4.0 International, CC-BY 4.0)
% You are free to copy and redistribute the material in any medium or format. You are free to remix, transform, and build upon the material for any purpose, even commercially. You must give appropriate credit, provide a link to the license, and indicate if changes were made. You may not apply legal terms or technological measures that legally restrict others from doing anything the license permits. No warranties are given.

\section{Zugfolge mit Fahrstraßen Aufgabe}

  \roles
    \begin{itemize}
      \item Spielleiter
      \item Triebfahrzeugführer eines beliebigen Zuges
      \item Triebfahrzeugführer eines anderen beliebigen Zuges
      \item Fahrdienstleiter
    \end{itemize}

  \material
    \begin{itemize}
      \item Gleise
      \item Weichen mit
      \begin{itemize}
          \item Fahrstraßenfelder
          \item Fahrstraßenzugschlussstellen
      \end{itemize}
      \item Mindestens zwei vollständige Blöcke mit Vorsignal, Hauptsignal und Signalzugschlusstelle
      \item zwei unterschiedliche Züge mit zugehörigen Fahrdynamikmodell, Zugschluss- und -spitzensignal
    \end{itemize}

  \setup
    \tikzfigure{challenge3_setup1.tikz}
    Die beiden Züge Stehen im Bahnhof A und sollen nacheinander in den Bahnhof B fahren. Weiche im abzweigenden Strang dürfen jeweils maximal mit \SI{80}{\kilo\metre\per\hour} befahren werden. Im durchgehenden Strang ist die Geschwindigkeit nicht begrenzt.

  \task
    \begin{enumerate}[label=\alph*)]
      \item Ergänzt die Infrastruktur mit Vorsignalen, Blocksignalen, Fahrstraßensignalen, Signalzugschlussstellen und Fahrstraßenzugschlussstellen!
      \item Wählt begründet den Zug aus, der als erster abfahren soll!
      \item Wie viele Runden dauert es bis der zweite Zug abfahren kann?
      \item Wie viele Runden dauert es bis beide Züge (Summe der Runden von Zug 1 und Zug 2) im Zielbahnhof angekommen sind?
    \end{enumerate}
