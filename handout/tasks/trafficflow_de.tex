%!TEX TS-program = pdflatexmk
%!TEX root = ../handbook.tex

% Copyright 2020 Martin Scheidt (Attribution 4.0 International, CC-BY 4.0)
% You are free to copy and redistribute the material in any medium or format. You are free to remix, transform, and build upon the material for any purpose, even commercially. You must give appropriate credit, provide a link to the license, and indicate if changes were made. You may not apply legal terms or technological measures that legally restrict others from doing anything the license permits. No warranties are given.

\section{Verkehrsfluss}

  \roles
    \begin{itemize}
      \item Spielleiter
      \item Triebfahrzeugführer eines beliebigen Zuges
      \item Triebfahrzeugführer eines anderen beliebigen Zuges
      \item Fahrdienstleiter
    \end{itemize}

  \material
    \begin{itemize}
      \item Gleise
      \item Mindestens drei vollständige Blöcke mit:
      \begin{itemize}
        \item Vorsignal,
        \item Hauptsignal und
        \item Signalzugschlusstelle
      \end{itemize}
      \item zwei beliebige Züge mit zugehörigen Fahrdynamikmodell, Zugschluss- und -spitzensignalen
    \end{itemize}

  \setup
    Die Strecke wird in einer Richtung betrieben. Auf der einen Seite brechen die Züge ein und auf der anderen Seite brechen sie aus. Die Infrastruktur vor und hinter der Strecke wird nicht berücksichtigt.

  \task
    \begin{enumerate}[label=\alph*)]
      \item Wie viele Runden werden benötigt vom Einbruch des ersten Zuges bis zum Verlassen des zweiten Zuges, wenn beide Züge behinderungsfrei fahren sollen und der schnelle vor dem langsamen Zug fährt?
      \item Wie viele Runden werden benötigt vom Einbruch des ersten Zuges bis zum Verlassen des zweiten Zuges, wenn beide Züge behinderungsfrei fahren sollen und der langsame vor dem schnellen Zug fährt?
    \end{enumerate}
